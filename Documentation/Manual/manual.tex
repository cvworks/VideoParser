\documentclass[twoside,letterpaper]{refart}
\usepackage{makeidx}
\usepackage{ifthen}
% ifthen wird vom Bild von N.Beebe gebraucht!

\def\bs{\char'134 } % backslash in \tt font.
\newcommand{\ie}{i.\,e.,}
\newcommand{\eg}{e.\,g..}
\DeclareRobustCommand\cs[1]{\texttt{\char`\\#1}}

%% Based on a TeXnicCenter-Template by Tino Weinkauf.
%%%%%%%%%%%%%%%%%%%%%%%%%%%%%%%%%%%%%%%%%%%%%%%%%%%%%%%%%%%%%

%%%%%%%%%%%%%%%%%%%%%%%%%%%%%%%%%%%%%%%%%%%%%%%%%%%%%%%%%%%%%
%% PDF-Informations
%%%%%%%%%%%%%%%%%%%%%%%%%%%%%%%%%%%%%%%%%%%%%%%%%%%%%%%%%%%%%
%%
%% ATTENTION: You need a main file to use this one here.
%%            Use the command "\input{filename}" in your
%%            main file to include this file.
%%

\pdfinfo{                               % Info dictionary of PDF output;
                                        % all keys are optional.
    /Author (Diego Macrini)
    /CreationDate (D:20101124011600)    % (D:YYYYMMDDhhmmss)
                                        % YYYY  year
                                        % MM    month
                                        % DD    day
                                        % hh    hour
                                        % mm    minutes
                                        % ss    seconds
                                        %
                                        % default: the actual date
                                        %
    /ModDate (D:20101124011600)         % ModDate is similar
    /Creator (TeX && TXC)               % default: "TeX"
    /Producer (pdfTeX)                  % default: "pdfTeX" + pdftex version
    /Title (VideoParser User's Manual)
    /Subject (Manual for using VideoParser)
    /Keywords (video parsing, object recognition)
}


\title{VideoParser User's Manual}
\author{Diego Macrini}

\date{}
\emergencystretch1em  %

\pagestyle{myfootings}
\markboth{Changing the layout with \textrm{\LaTeX}}%
         {Changing the layout with \textrm{\LaTeX}}

\makeindex 

\setcounter{tocdepth}{2}

\begin{document}

\maketitle

\begin{abstract}
        This document describes the capabilities of the 
        \texttt{refart} and \texttt{refrep} classes for \LaTeXe. 
        These classes do not work with \LaTeX\ 2.09. They contain some 
        improvements over the original \texttt{refman} style which may result 
        in different output and minor incompatibilities, but make refman 
        work with paper sizes other than ISO A4, which I consider an 
        improvement.
\end{abstract}


This manual is an addition to chapter~5 (``Designing It Yourself'') of 
the \LaTeX-manual by Leslie Lamport.  It was originally written in 
1988 by Hubert Partl and updated by me during the development of 
\texttt{refman}~2.0.  The translation into English was done in summer 
1998, almost 10 years after the initial release in German.

\tableofcontents

\newpage


%%%%%%%%%%%%%%%%%%%%%%%%%%%%%%%%%%%%%%%%%%%%%%%%%%%%%%%%%%%%%%%%%%%%

\section{Introduction}


\printindex

\end{document}
